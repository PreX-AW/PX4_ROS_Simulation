\documentclass{article}
\usepackage[UTF8]{ctex}
\usepackage{minted}
\usepackage{appendix}
\usepackage{hyperref}
\usepackage{geometry}

\geometry{a4paper}
\geometry{centering}
\geometry{left=2cm}
\geometry{right=2cm}
\geometry{top=1cm}
\geometry{bottom=1cm}



\renewcommand\contentsname{\textbf{Content}}
\newcommand\mintbash[1]{\mintinline[breakanywhere,breaklines]{bash}| #1 |}
\newcommand\bash[1]{\mintinline[breakanywhere,breaklines]{bash}| #1 |}


\begin{document}
\author{Wei Hu}
\date{\today}
\title{\textbf{ROS PX4 仿真教程}}

\maketitle
%\section{Ubuntu安装}
%使用\href{URLhttps://rufus.ie/en/}{Rufus}制作系统镜像盘,注意新一点的电脑选择GPT硬盘格式,UEFI启动,之后制作镜像启动盘即可

\section{Ubuntu下ROS开发的配置}
首先应使用命令\bash{sudo apt update && sudo apt upgrade}更新系统软件源以及软件,之后在进行以下操作
\subsection{Ubuntu系统配置}
\subsubsection{系统时间设置}
完成双系统安装完成之后,\texttt{Ubuntu}系统与原有的\texttt{Windows}系统的时间会发生错乱
,\href{URLhttps://zhuanlan.zhihu.com/p/492885761}{双系统下的时间错乱},
总而言之就是两个系统对于主板上的\mintinline{bash}| BIOS |时间认识不一致
,解决方法如下命令
\begin{minted}{bash}
    timedatectl status 
    # 查看系统时间状态

    timedatectl set-local-rtc 1 
    # 设置Ubuntu系统时间与BIOS同步
    
    timedatectl status 
    # 此条命令可以不用输入,也是用于查看修改后的系统时间状态
    # 会有一个Warning输出.
\end{minted}
\subsubsection{中文输入法}
进入设置,转到地区与语言,点击管理已下载的语言,会有弹窗提示更新,更新完之后重启电脑,之后再次进入设置转到地区与语言
在输入源中添加\bash{Chinese(Intelligent Pinyin)}中文应该是智能拼音.
\subsection{相关软件安装}
这一部分是为了更好的使用\texttt{ROS},我们需要下载以下几个软件
\begin{itemize}
    \item [1.]\href{https://code.visualstudio.com/download}{Visual Studio code},
          其中很多插件可以使得编写代码更为方便快捷.\\插件推荐:C/C++,Python,ROS,CMake,One Dark Pro,
    \item [2.]\href{https://www.google.com/chrome/}{Google Chrome},
          使用\texttt{Google}的话需要
          自己的电脑有科学上网的能力,如果不能够科学上网,
          可以使用\texttt{Ubuntu}自带的\texttt{Firefox}浏览器,在设置里将浏览器修改为\texttt{Bing}就可以了.
    \item [3.]\href{https://git-scm.com/downloads}{Git},一般而言,
          如果正确安装t\texttt{Ubuntu}的话,在命令行输入\texttt{Git}
          就会有相关内容显示,关于\texttt{Git}相关的教程请看附录
    \item [4.]\href{https://www.vim.org/download.php}{Vim},一个可以集成到终端中的代码编辑器,不用离开终端,
          只需输入命令就可以在终端直接编辑文件.
          当然了,在\texttt{Ubuntu}下可以直接使用命令行输入命令
          \mintinline{bash}{sudo apt install vim},但是令人头疼的也是它的上手门槛,
          相关简单操作以及\texttt{Vim}配置见附录.
\end{itemize}

\section{ROS安装}


\href{URLhttps://wiki.ros.org/ROS/Introduction}{\textbf{ROS介绍}}
\\

\href{https://wiki.ros.org/noetic/Installation/Ubuntu}{\textbf{ROS安装}}
\begin{itemize}
    \item ROS安装最好是在使用科学上网的前提下进行,否则会遇到各种各样的报错,在科学上网的前提下,按照官网的安装指令,一步一步的执行即可
    \item \href{https://fishros.org.cn/forum/topic/20/%E5%B0%8F%E9%B1%BC%E7%9A%84%E4%B8%80%E9%94%AE%E5%AE%89%E8%A3%85%E7%B3%BB%E5%88%97?lang=en-US}{fishros}一键安装,使用命令:\bash{wget http://fishros.com/install -O fishros && . fishros}
\end{itemize}


\section{PX4仓库克隆及编译}
在正式克隆\texttt{PX4}仓库之前需要保证以下几点
\begin{itemize}
    \item 成功访问\href{URLhttps://github.com/}{Github},并且注册一个自己的\texttt{Github}账户.
    \item 观看过附录的\textbf{Linux终端常用命令}
    \item 观看过附录的\textbf{Git简易教程}
\end{itemize}
打开终端,切换到合适的目录下输入命令\mintbash{git clone https://github.com/PX4/PX4-Autopilot.git}并进入\mintbash{ PX4-Autopilot }目录,输入命令\mintbash{bash ./Tools/setup/ubuntu.sh}
,等待命令执行结束之后,输入命令\mintbash{sudo reboot}重启电脑.
\\
等待重启完毕之后,进入\mintbash{PX4-Autopilot}目录,输入命令\mintbash{make px4_sitl gazebo-classic}即可启动仿真界面,并在刚刚输入命令的界面再次输入\mintbash{commander takeoff}即可实现飞机起飞
输入命令\mintbash{commander land}\\可实现飞机降落
\subsection{PX4仿真与ROS通讯}
需要下载\mintbash{mavros}功能包,打开终端输入命令
\begin{minted}{bash}
    sudo apt install ros-noetic-mavros ros-noetic-mavros-extras ros-noetic-mavros-msgs
\end{minted}
根据官网的\href{https://docs.px4.io/main/en/ros/mavros_offboard_cpp.html}{\bash{demo}}修改对应的配置文件,启动节点即可实现\bash{ROS 与 PX4通讯}
以下是启动无人机正方形飞行的命令
\begin{appendices}
    \section{Linux终端常用命令}
    \begin{itemize}
        \item 打开终端:\mintinline{bash}|Ctrl + Alt + t|
        \item 创建目录:\bash{mkdir <folder_name>}
        \item 在终端下切换目录:\mintinline{bash}| cd <path> |\\
              回到上一级目录\mintinline{bash}| cd .. |\\
              \mintinline{bash}| . |表示当前目录,所以也可以使用相对于当前目录来使用\mintinline{bash}| cd |命令来切换目录,
              \\如果当前目录的父目录\bash{parent}有个叫\bash{parent_test}的目录,
              则可以使用命令\mintinline{bash}| cd ../parent_test |来切换到\bash{parent_test}目录\\
              如果当前目录的子目录\mintinline{bash}| child |下有个\bash{child_test}的目录,则可以使用命令\bash{cd ./child/child_test}來切換到\bash{child_test}目录.
              \item 执行文件:使用命令\bash{./<file_name>}
    \end{itemize}
    \section{Git 简易教程}
    下載git之后,需要配置自己的用户名以及用户邮箱,使用\mintinline{bash}| Ctrl + Alt + t |打开终端,输入以下命令:
    \begin{minted}{bash}
        git config --global user.name "your name"
        git config --global user.email "your email"
    \end{minted}
    \subsection{Github 简易使用}
    如果是自己要在\mintbash{Github}创建仓库或者团队开发需求,首先要在自己的机器上
    生成用于访问\mintbash{Github}\\的\mintbash{SSH}密钥,在终端输入\mintbash{ssh-keygen -r rsa -C "youremail"},如果只是个人使用的话,在输入万命令之后一路回车或者输入\mintbash{yes}即可
    生成的密钥一般会存放在用户目录下的隐藏文件夹\mintbash{.ssh}下,拷贝该目录下一个名为\mintbash{id_ras.pub}文件中的内容,登陆\mintbash{Github},
    转到\mintbash{Settings}中的\mintbash{SSH and GPG keys},点击\mintbash{New},将\mintbash{id_ras.pub}中的内容粘贴进去,
    点击\mintbash{Add SSH key},之后点击\mintbash{Save}即可.
    \subsubsection{克隆自己仓库}
    使用命令
    \begin{minted}{bash}
        git clone https://github.com/yourname/yourrepository
    \end{minted}
    即可克隆自己的仓库,克隆到本地之后就可以开始编辑工作,
    要想将本地的修改上传到\mintbash{Github}仓库中,
    需要在本地仓库中执行\mintbash{git add .},然后执行\mintbash{git commit -m "commit message"},
    最后执行\mintbash{git push}即可上传到\mintbash{Github}仓库中,如果是第一次推送的话,需要
    添加参数\mintbash{git push -u origin main},之后推送直接使用\mintbash{git push}即可。
    如果后续在\bash{Github}上修改了仓库里的代码,本地机器上的代码与\bash{Github}代码不同,则使用命令\bash{git pull}即可拉取远程仓库的代码。
    
    \subsubsection{克隆他人仓库}对于公开仓库,直接使用以下命令即可克隆他人仓库
    \begin{minted}{bash}
    git clone https://github.com/othername/otherrepository
\end{minted}
    如果是私有仓库的话,需要在\mintbash{Github}中添加自己的\mintbash{SSH}密钥,然后使用下列命令,即可克隆他人仓库,克隆到本地之后就可以开始编辑工作。
    \begin{minted}{bash}
    git clone git@github.com:othername/otherrepository
    \end{minted}
    如果后期他人仓库中的内容有所改动,则只需要输入命令\bash{git pull}即可拉取他人仓库更新后的源码.
    \section{Vim 简易教程及其配置}
    \subsection{Normal模式}
    \mintbash{Vim}有多种模式,一般使用命令打开文件进入的是\mintbash{Normal}模式,可以使用\mintbash{hjkl}来进行光标的运动,
    \mintbash{h&l}分别代表光标左右运动,\mintbash{k&j}分别代表光标上下行运动
    然后使用\mintbash{gg}命令跳转到\emph{文件开头},使用\mintbash{dd}命令删除\emph{光标所在行},使用\mintbash{yy}复制光标所在行,使用\mintbash{p}则是粘贴之前所复制的行.、
\subsection{Insert模式}
在\mintbash{Normal}模式下输入\bash{i}进入光标所在位置开始编辑,输入\bash{a}进入光标之后位置开始编辑(更加符合我们平时编辑的模式),
输入\bash{o}从当前光标所在行新起一行开始编辑,输入\bash{shift + o}从当前光标所在行向上新起一行开始编辑。

% \subsection{Vim 配置}
%    打开终端切换到家目录下,命令:\bash{cd ~}
%    输入命令\bash{git clone https://gitee.com/mirrorvim/vim-fast.git}
%    进入\bash{vim-fast}目录,使用命令\bash{sudo ./install.sh}等待完成之后,在当前目录下使用命令下的\bash{shell}目录
\end{appendices}

\end{document}