\documentclass[UTF8]{article}
\usepackage{ctex}
\usepackage{minted}
\usepackage{appendix}
\usepackage{hyperref}
\renewcommand\contentsname{\textbf{Content}}


\begin{document}
\author{Wei Hu}
\date{\today}
\title{\textbf{ROS PX4 仿真教程}}
\maketitle
\tableofcontents
\section{Ubuntu安装}
\section{Ubuntu下ROS开发的配置}
\subsection{相关软件安装}
这一部分是为了更好的使用\texttt{ROS},我们需要下载以下几个软件
\begin{itemize}
    \item [1.]\href{https://code.visualstudio.com/download}{Visual Studio code},
          其中很多插件可以使得编写代码更为方便快捷。插件推荐:C/C++,Cmake,Python,ROS.
    \item [2.]\href{https://www.google.com/chrome/}{Google Chrome},
          使用\texttt{Google}的话需要
          自己的电脑有科学上网的能力,如果不能够科学上网,
          可以使用\texttt{Ubuntu}自带的\texttt{Firefox}浏览器,在设置里将浏览器修改为\texttt{Bing}就可以了。
    \item [3.]\href{https://git-scm.com/downloads}{git},一般而言,
          如果正确安装t\texttt{Ubuntu}的话,在命令行输入\texttt{Git}
          就会有相关内容显示,关于\texttt{Git}相关的教程请看附录
    \item [4.]\href{https://www.vim.org/download.php}{Vim},一个可以集成到终端中的代码编辑器,不用离开终端,
          只需输入命令就可以在终端直接编辑文件。
          当然了,在\texttt{Ubuntu}下可以直接使用命令行输入命令
          \mintinline{bash}{sudo apt install vim},但是令人头疼的也是它的上手门槛,
          相关简单操作以及\texttt{Vim}配置见附录。
\end{itemize}
\subsection{Ubuntu系统配置}
\subsubsection{系统时间设置}
完成双系统安装完成之后,\texttt{Ubuntu}系统与原有的\texttt{Windows}系统的时间会发生错乱
,\href{URLhttps://zhuanlan.zhihu.com/p/492885761}{双系统下的时间错乱},总而言之就是两个系统对于主板上的`BIOS'时间认识不一致
,解决方法如下命令
\begin{minted}{bash}
    timedatectl status 
    # 查看系统时间状态

    timedatectl set-local-rtc 1 
    # 设置Ubuntu系统时间与BIOS同步
    
    timedatectl status 
    # 此条命令可以不用输入,也是用于查看修改后的系统时间状态,会有一个Warning输出。
\end{minted}
\subsubsection{中文输入法}
进入设置,
\section{ROS安装}

\href{URLhttps://wiki.ros.org/ROS/Introduction}{ROS介绍}
\href{https://wiki.ros.org/noetic/Installation/Ubuntu}{ROS安装}

\section{PX4仓库克隆及编译}
在正式克隆\texttt{PX4}仓库之前需要保证以下几点
\begin{itemize}
    \item 成功访问\href{URLhttps://github.com/}{Github},并且注册一个自己的\texttt{Github}账户。
    \item 观看过附录的\textbf{Linux终端常用命令}
    \item 观看过附录的\textbf{Git简易教程}
\end{itemize}
\begin{appendices}
    \renewcommand{\thesection}{\Alph{section}}
    \section{附录标题}
    \section{Linux终端常用命令}
    \begin{itemize}
        \item 打开终端:\mintinline{bash}|Ctrl + Alt + t|
        \item 在终端下切换目录:\mintinline{bash}| cd <path> |\\
              回到上一级目录\mintinline{bash}| cd .. |\\
              \mintinline{bash}| . |表示当前目录,所以也可以使用相对于当前目录来使用\mintinline{bash}| cd |命令来切换目录,如果当前目录的父目录有个叫\mintinline{bash}| test |的目录,
              则可以使用命令\mintinline{bash}| cd ../test |来切换到\mintinline{bash}| test |目录\\
              如果当前目录下有个子目录\mintinline{bash}| child |下
    \end{itemize}
    \section{Git 简易教程}
    \section{Vim 简易教程及其配置}
\end{appendices}

\end{document}